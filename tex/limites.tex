\section{Limites}

	\subsection{Sucessões ou Sequências \cite{morettin}}

	\subsection{Convergência de Sucessões \cite{morettin}}

	\subsection{Limite de Funções \cite{morettin}}

	\subsection{Propriedades Operatórias \cite{somatematica} \cite{ventura}}

		As propriedades operatórias permitem achar os limites de somas, diferenças, produtos, quocientes e outros mais de funções elementares.
		
		Para funções $f$ e $g$ com limites para $x \to a$, $\lim \limits_{x \to a} f(x) = L$ e $\lim \limits_{x \to a} g(x) = M$, desde que $(L, M \neq \infty )$, temos:

		\medskip

		\begin{enumerate}[label=(P\arabic*)]

			\item {\LARGE $\lim \limits_{x \to a} [f(x) \pm g(x)] = \lim \limits_{x \to a} f(x) \pm \lim \limits_{x \to a} g(x)$}

			\medskip

			\textbf{Exemplo}: $\lim \limits_{x \to 1} [x^{2} + 3x^{3}] = \lim \limits_{x \to 1} x^{2} + \lim \limits_{x \to 1} 3x^{3} = 1 + 3 = 4$ ;

			\item {\LARGE $\lim \limits_{x \to a} [f(x) \times g(x)] = \lim \limits_{x \to a} f(x) \times \lim \limits_{x \to a} g(x)$}

			\medskip

			\textbf{Exemplo}: $\lim \limits_{x \to \pi} [3x^{3} \times \cos x] = \lim \limits_{x \to \pi} 3x^{3} \times \lim \limits_{x \to \pi} \cos x = 3\pi ^{3} \times cos \pi = 3 \pi ^{3} \times (-1) = - 3 \pi ^{3}$ ;

			\item {\LARGE $\lim \limits_{x \to a} \cfrac{f(x)}{g(x)} = \cfrac{\lim \limits_{x \to a} f(x)}{\lim \limits_{x \to a} g(x)}$, desde que $\lim \limits_{x \to a} g(x) \neq 0$}

			\medskip

			\textbf{Exemplo}: $\lim \limits_{x \to 0} \cfrac{\cos x}{x^{2} + 1} = \cfrac{\lim \limits_{x \to 0} \cos x}{\lim \limits_{x \to 0} x^{2} + 1} = \cfrac{\cos 0}{0^{2} + 1} = \cfrac{1}{1} = 1$ ;

			\item {\LARGE $\lim \limits_{x \to a} f(x)^{n} = \left( \lim \limits_{x \to a} f(x) \right) ^{n}$, desde que $n \in \mathbb{N}^{*}$}

			\medskip

			\textbf{Exemplo}: $\lim \limits_{x \to 1} (x^{2} + 3)^{2} = \left[ \lim \limits_{x \to 1} (x^{2} + 3)^{2} \right] = (1 + 3)^{2} = 16$ ;

			\item {\LARGE $\lim \limits_{x \to a} \sqrt[n]{f(x)} = \sqrt[n]{\lim \limits_{x \to a} f(x)}$, desde que $n \in \mathbb{N}^{*}$ e $f(x) > 0$ (se $f(x) \leq 0$ $n$ é ímpar)}

			\medskip

			\textbf{Exemplo}: $\lim \limits_{x \to 2} \sqrt{x^{3} + x^{2} - 1} = \sqrt{\lim \limits_{x \to 2} x^{3} + x^{2} - 1} = \sqrt{2^{3} + 2^{2} - 1} = \sqrt{11}$ ;

			\item {\LARGE $\lim \limits_{x \to a} (\ln f(x)) = \ln \left[ \lim \limits_{x \to a} f(x) \right]$, desde que $\lim \limits_{x \to a} f(x) > 0$}

			\medskip

			\textbf{Exemplo}: $\lim \limits_{x \to e} (\ln x^{2}) = \ln \left[ \lim \limits_{x \to e} x^{2} \right] = \ln e^{2} = 2 \ln e = 2 \times 1 = 2$ ;

			\item {\LARGE $\lim \limits_{x \to a} \sin (f(x)) = \sin \left( \lim \limits_{x \to a} f(x) \right)$}

			\medskip

			\textbf{Exemplo}: $\lim \limits_{x \to 1} \sin (x^{2} + 3x) = \sin \left[ \lim \limits_{x \to 1} (x^{2} +3x) \right] = \sin 4$ ;

			\item {\LARGE $\lim \limits_{x \to a} e^{f(x)} = e^{\lim \limits_{x \to a} f(x)}$}

			\medskip

			\textbf{Exemplo}: $\lim \limits_{x \to 1} e^{x^{2} + 3x} = e^{\lim \limits_{x \to 1} x^{2} +3x} = e^{4}$ .

		\end{enumerate}

	\subsection{Formas Indeterminadas \cite{morettin}}

	\subsection{Limites Infinitos \cite{morettin}}

	\subsection{Limites nos Extremos do Domínio \cite{morettin}}

	\subsection{Continuidade de uma Função \cite{morettin}}

	\subsection{Assíntotas Verticais e Horizontais \cite{morettin}}

	\subsection{Limite Exponencial Fundamental \cite{morettin}}