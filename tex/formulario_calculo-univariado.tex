\section{Cálculo Univariado}

	\subsection{Limites}

		\begin{longtable}{
		@{}
		C{.475\textwidth} |
		C{.5\textwidth}
		@{}}
		
			\toprule
			%------------------------------
			\textbf{Continuidade num Ponto} & \textbf{Teorema do Confronto}
			%------------------------------
			\tabularnewline
			\midrule
			%------------------------------
			{\large $\lim \limits_{x \to b^{+}} f(x) = \lim \limits_{x \to b^{-}} = f(b)$} & {\large \begin{tabular}[c]{@{}c@{}} se $g(x) \geq f(x) \leq h(x)$, \\ $\lim \limits_{x \to a} g(x) = \lim \limits_{x \to a} h(x) = \lim \limits_{x \to a} f(x) = L$ \end{tabular}}
			%------------------------------
			\tabularnewline
			\midrule
			%------------------------------
			\multicolumn{2}{c}{\textbf{Limites nos Extremos}}
			%------------------------------
			\tabularnewline
			\midrule
			%------------------------------
			\multicolumn{2}{c}{{\large \begin{tabular}[c]{@{}c@{}} $\lim \limits_{x \to \infty} (2x^{3} + 4x^{2} - 5x + 9) =$ $\lim \limits_{x \to \infty} 2x^{3} \left(1 + \cfrac{2}{x} - \cfrac{5}{2x^{2}} + \cfrac{9}{2x^{3}}) \right) =$ \\ $\lim \limits_{x \to \infty} 2x^{3}$ \end{tabular}}}
			%------------------------------
			\tabularnewline
			\midrule
			%------------------------------
			\multicolumn{2}{c}{\textbf{Propriedades Operatórias}}
			%------------------------------
			\tabularnewline
			\midrule
			%------------------------------
			\multicolumn{2}{c}{{\large$\lim \limits_{x \to a} [f(x) \pm g(x)] = \lim \limits_{x \to a} f(x) \pm \lim \limits_{x \to a} g(x)$}}
			%------------------------------
			\tabularnewline
			\midrule
			%------------------------------
			\multicolumn{2}{c}{{\large $\lim \limits_{x \to a} \cfrac{f(x)}{g(x)} = \cfrac{\lim \limits_{x \to a} f(x)}{\lim \limits_{x \to a} g(x)}$, desde que $\lim \limits_{x \to a} g(x) \neq 0$}}
			%------------------------------
			\tabularnewline
			\midrule
			%------------------------------
			\multicolumn{2}{c}{{\large $\lim \limits_{x \to a} f(x)^{n} = \left( \lim \limits_{x \to a} f(x) \right) ^{n}$, desde que $n \in \mathbb{N}^{*}$}}
			%------------------------------
			\tabularnewline
			\midrule
			%------------------------------
			\multicolumn{2}{c}{{\large \begin{tabular}[c]{@{}c@{}} $\lim \limits_{x \to a} \sqrt[n]{f(x)} = \sqrt[n]{\lim \limits_{x \to a} f(x)}$, \\ desde que $n \in \mathbb{N}^{*}$ e $f(x) > 0$ (se $f(x) \leq 0$ $n$ é ímpar)\end{tabular}}}
			%------------------------------
			\tabularnewline
			\midrule
			%------------------------------
			\multicolumn{2}{c}{{\large $\lim \limits_{x \to a} (\ln f(x)) = \ln \left[ \lim \limits_{x \to a} f(x) \right]$, desde que $\lim \limits_{x \to a} f(x) > 0$}}
			%------------------------------
			\tabularnewline
			\midrule
			%------------------------------
			\multicolumn{2}{c}{{\large $\lim \limits_{x \to a} \sin (f(x)) = \sin \left( \lim \limits_{x \to a} f(x) \right)$}}
			%------------------------------
			\tabularnewline
			\midrule
			%------------------------------
			\multicolumn{2}{c}{{\large $\lim \limits_{x \to a} e^{f(x)} = e^{\lim \limits_{x \to a} f(x)}$}}
			%------------------------------
			\tabularnewline
			\bottomrule

		\end{longtable}
		
	\subsection{Derivadas}

		\begin{longtable}{
		@{}
		C{1\textwidth} 
		@{}}
		
			\toprule
			%------------------------------
			\textbf{Derivada por Definição}
			%------------------------------
			\tabularnewline
			\midrule
			%------------------------------
			{\large $\lim \limits_{\Delta x \to 0} \cfrac{\Delta f}{\Delta x} = \lim \limits_{\Delta x \to 0} \cfrac{f(x_{0} + \Delta x) - f(x_{0})}{\Delta x}$}
			%------------------------------
			\tabularnewline
			\midrule
			%------------------------------
			\textbf{Equação Reduzida da  Reta}
			%------------------------------
			\tabularnewline
			\midrule
			%------------------------------
			{\large $y = mx + n$ \hspace{1cm} $m = \cfrac{\Delta f}{\Delta x} \approx f'(x)$ \hspace{1 cm} $y - y_{0} = m(x - x_{0})$}
			%------------------------------
			\tabularnewline
			\midrule
			%------------------------------
			\textbf{Funções Elementares}
			%------------------------------
			\tabularnewline
			\midrule
			%------------------------------
			{\large $f(x) = c$ \hspace{1cm} $f'(x) = 0$}
			%------------------------------
			\tabularnewline
			\midrule
			%------------------------------
			{\large $f(x) = x^{n}$ \hspace{1cm} $f'(x) = n \times x^{n-1} \ , (x >0)$}
			%------------------------------
			\tabularnewline
			\midrule
			%------------------------------
			{\large $f(x) = \ln x$\hspace{1cm} $f'(x) = \cfrac{1}{x}$ \ , $(x > 0)$}
			%------------------------------
			\tabularnewline
			\midrule
			%------------------------------
			{\large $f(x) = \sin x$ \hspace{1cm} $f'(x) = \cos x$ \ , $(x \in \mathbb{R})$}
			%------------------------------
			\tabularnewline
			\midrule
			%------------------------------
			{\large $f(x) = \cos x$ \hspace{1cm} $f'(x) = \sin x$ \ , $(x \in \mathbb{R})$}
			%------------------------------
			\tabularnewline
			\midrule
			%------------------------------
			\textbf{Propriedades Operatórias}
			%------------------------------
			\tabularnewline
			\midrule
			%------------------------------
			{\large $f(x) = k \times g(x)$ \hspace{1cm} $f'(x) = k \times g'(x)$ \ , $k = \text{constante}$}
			%------------------------------
			\tabularnewline
			\midrule
			%------------------------------
			{\large $f(x) = u(x) \pm v(x)$ \hspace{1cm} $f'(x) = u'(x) \pm v'(x)$}
			%------------------------------
			\tabularnewline
			\midrule
			%------------------------------
			{\large $f(x) = u(x) \times v(x)$ \hspace{1cm} $f'(x) = u'(x) \times v(x) + u(x) \times v'(x)$}
			%------------------------------
			\tabularnewline
			\midrule
			%------------------------------
			{\large $f(x) = \cfrac{u(x)}{v(x)}$ \hspace{1cm} $f'(x) = \cfrac{u'(x) \times v(x) - u(x) \times v'(x)}{[v(x)]^{2}}$}
			%------------------------------
			\tabularnewline
			\midrule
			%------------------------------
			{\large $f(x) = \cfrac{1}{v}$ \hspace{1cm} $f'(x) = \cfrac{v'}{v^{2}}$}
			%------------------------------
			\tabularnewline
			\midrule
			%------------------------------
			\textbf{Funções Especiais (Composta - Regra da Cadeia, Exponencial, Exponencial Geral}
			%------------------------------
			\tabularnewline
			\midrule
			%------------------------------
			{\large $\cfrac{d}{dx} f(g(x)) = f'(g(x)) \times g'(x)$}
			%------------------------------
			\tabularnewline
			\midrule
			%------------------------------
			{\large $f(x) = a^{x}$ \hspace{1cm} $f'(x) = a^{x} \times \ln a$ \ , $\forall x \in \mathbb{R} \mid a > 0$ \text{e} $a \neq 1$}
			%------------------------------
			\tabularnewline
			\midrule
			%------------------------------
			{\large \begin{tabular}[c]{@{}c@{}} $f(x) = u(x)^{v(x)} \implies f(x) = x^{x}$ \\ $\ln f(x) = x \times \ln x$ \ , ($\log_{a} M^{\alpha} = \alpha \times \log_{a} M$) \\ $\cfrac{1}{f(x)} \times f'(x) = 1 \times \ln x + x \times \cfrac{1}{x}$ \\ $f'(x) = f(x) \times [\ln x +1] \implies f'(x) = x^{x} \times [\ln x + 1]$ \end{tabular}}
			%------------------------------
			\tabularnewline
			\midrule
			%------------------------------
			\textbf{Diferencial de uma Função}
			%------------------------------
			\tabularnewline
			\midrule
			%------------------------------
			{\large $df = f'(x_{0}) \times \Delta x$ \hspace{1cm} $df \approx \Delta f$ \ para pequenos valores de $\Delta x$}
			%------------------------------
			\tabularnewline
			\midrule
			%------------------------------
			\textbf{Derivadas de $2^{a}$, $3^{a}$, $\dots$ Ordem}
			%------------------------------
			\tabularnewline
			\midrule
			%------------------------------
			{\large $f'(x) \ f''(x) \ f'''(x) \ f^{(4)}(x)$}
			%------------------------------
			\tabularnewline
			\bottomrule

		\end{longtable}
		
	\subsection{Funções Financeiras/Administrativas}

		\begin{longtable}{
		@{}
		C{1\textwidth} 
		@{}}

			\toprule
			%------------------------------
			{\large $R(P \ ou \ Q) = P \times Q$ \hspace{1cm} $L(P \ ou \ Q) = R(P ou Q) - C (P ou Q)$}
			%------------------------------
			\tabularnewline
			\midrule
			%------------------------------
			{\large $C_{mg}(x) = C'(x)$ \hspace{1cm} $R_{mg}(x) = R'(x)$ \hspace{1cm}  normalmente $\Delta x = 1$}
			%------------------------------
			\tabularnewline
			\midrule
			%------------------------------
			{\large Prop. Marginal a Consumir(C) $\rightarrow p^{c}_{mg} = C'(y)$ \hspace{1cm} $y =$ renda disponível}
			%------------------------------
			\tabularnewline
			\midrule
			%------------------------------
			{\large Prop. Marginal a Poupar(S) $\rightarrow  p^{c}_{mg} = S'(y)$ \hspace{1cm} $y =$ renda disponível}
			%------------------------------
			\tabularnewline
			\midrule
			%------------------------------
			{\large Produtividade Marginal $\rightarrow P_{mg}(x) = P'(x)$}
			%------------------------------
			\tabularnewline
			\bottomrule

		\end{longtable}

	\subsection{Elasticidade - Função Oferta e Demanda}

		\begin{longtable}{
		@{}
		C{1\textwidth} 
		@{}}

			\toprule
			%------------------------------
			{\large \begin{tabular}[c]{@{}c@{}} Função Demanda $\rightarrow \cfrac{dx}{dp} < 0$ \hspace{1cm} Função Oferta $\rightarrow \cfrac{dx}{dp} > 0$ \\ Ponto de Equilíbrio $\rightarrow p_{d} = p_{o}$ \end{tabular}}
			%------------------------------
			\tabularnewline
			\midrule
			%------------------------------
			{\large \begin{tabular}[c]{@{}c@{}} $\cfrac{\Delta p}{p_{0}} =$ variação percentual no preço \\ $\cfrac{\Delta x}{x_{0}} =$ variação percentual na quantidade \end{tabular}}
			%------------------------------
			\tabularnewline
			\midrule
			%------------------------------
			{\large $e = \begin{vmatrix} \lim \limits_{\Delta p \to 0} \cfrac{\cfrac{\Delta x}{x_{0}}}{\cfrac{\Delta p}{p_{0}}} \end{vmatrix} = \cfrac{p_{0}}{x_{0}} \times \begin{vmatrix} \lim \limits_{\Delta p \to 0} \cfrac{\Delta x}{\Delta p}\end{vmatrix} = \cfrac{p_{0}}{x_{0}} \times \begin{vmatrix} \cfrac{dx}{dp}\end{vmatrix}$}
			%------------------------------
			\tabularnewline
			\midrule
			%------------------------------
			{\large \begin{tabular}[c]{@{}c@{}}
			
			$e > 1 \rightarrow$ elástica \hspace{0.5cm} $0 < e > 1 \rightarrow$ inelástica \hspace{0.5cm} $e = 1 \rightarrow$ elasticidade unitária \\
			
			*Não se aplica em elasticidade cruzada.
			
			\end{tabular}}
			%------------------------------
			\tabularnewline
			\bottomrule

		\end{longtable}

	\subsection{Estudo de Funções Univariadas}

		\begin{longtable}{
		@{}
		C{1\textwidth} 
		@{}}

			\toprule
			%------------------------------
			{\large Pontos críticos $\rightarrow f'(x) = 0$ ou $f'(x) = \nexists$ ou indefinido}
			%------------------------------
			\tabularnewline
			\midrule
			%------------------------------
			{\large \begin{tabular}[c]{@{}c@{}} Ponto de Mínimo (convexidade $\bigcup$ ) $\rightarrow f''(x) > 0 \ \forall x \in \ ]a, b[$ \\ Ponto de Máximo (concavidade $\bigcap$ ) $\rightarrow f''(x) < 0 \ \forall x \in \ ]a, b[$ \\ Ponto de Inflexão $\rightarrow f''(x) = 0 \ \forall x \in \ ]a, b[$ \end{tabular}}
			%------------------------------
			\tabularnewline
			\midrule
			%------------------------------
			\textbf{Estudo Completo de uma Função}
			%------------------------------
			\tabularnewline
			\midrule
			%------------------------------
			{\large \begin{enumerate}[label=(\arabic*)] \item Determinação do domínio; \item Determinação das intersecções com os eixos, quando possível; \item Determinação dos intervalos de crescimento e decrescimento e de possíveis pontos de máximo e mínimo; \item Determinação dos intervalos em que a função é côncava para cima ou para baixo e de possíveis pontos de inflexão; \item Determinação dos limites nos extremos do domínio e de possíveis assíntotas; \item Determinação dos limites laterais nos pontos de descontinuidades (quando houver e possíveis assíntotas). \end{enumerate}}
			%------------------------------
			\tabularnewline
			\bottomrule

		\end{longtable}
		
	\subsection{Integrais}

		\begin{longtable}{
		@{}
		C{1\textwidth} 
		@{}}

			\toprule
			%------------------------------
			\textbf{Integral Indefinida}
			%------------------------------
			\tabularnewline
			\midrule
			%------------------------------
			{\large $\int g(x) dx = f(x) + c$}
			%------------------------------
			\tabularnewline
			\midrule
			%------------------------------
			\textbf{Funções Elementares}
			%------------------------------
			\tabularnewline
			\midrule
			%------------------------------
			{\large \begin{tabular}[c]{@{}c@{}}
			
    			$\int x^{n} dx = \cfrac {x^{n + 1}}{n+1} + c$ , $\forall n \in \mathbb{Z} \mid n \neq -1$ \\
    			
    			$\int \cfrac{1}{x} \ dx = \ln (-x) + c$ \\
    			
    			$\int x^{a} \ dx = \cfrac{x^{\alpha + 1}}{\alpha + 1} + c$ , $(\alpha \neq -1) \ (x > 0)$) \\
    			
    			$\int \cos xdx = \sin x + c$ , pois a derivada de $\sin x$ é $\cos x$ \\
    			
    			$\int \sin xdx = - \cos x + c$ , pois a derivada de $- \cos x$ é $\sin x$ \\
    			
    			$\int e^{x} \ dx = e^{x} + c$ , pois a derivada de $e^{x} $ é $e^{x}$ \\
    			
    			$\int \cfrac{1}{1 + x^{2}} \ dx = \arctan x + c$ , pois a derivada de $\arctan x$ é $\cfrac{1}{1+x^{2}}$ \\
    			
    			$\int \cfrac{1}{\sqrt{1 - x^{2}}} \ dx = \arcsin x + c$ , \\
    			
    			pois a derivada de $\arcsin x$ é $\cfrac{1}{\sqrt{1 - x^{2}}}$ , para $-1 < x < 1$
			
			\end{tabular}}
			%------------------------------
			\tabularnewline
			\midrule
			%------------------------------
			\textbf{Propriedades Operatórias}
			%------------------------------
			\tabularnewline
			\midrule
			%------------------------------
			{\large \begin{tabular}[c]{@{}c@{}}
			
    			$\int [ f_{1} (x) \pm f_{2} (x) ] \ dx = \int f_{1} (x) \ dx \pm \int f_{2} (x) \ dx$ \\
    			
    			$\int c \cdot f(x) \ dx = c \cdot \int f(x) \ dx$
			
			\end{tabular}}
			%------------------------------
			\tabularnewline
			\midrule
			%------------------------------
			\textbf{Integral Definida}
			%------------------------------
			\tabularnewline
			\midrule
			%------------------------------
			{\large $ \int_{a}^{b} f(x) \ dx = g(b) - g(a) = \lim \limits_{x \to b-} [g(x)] - \lim \limits_{x \to a+} [g(x)] $}
			%------------------------------
			\tabularnewline
			\midrule
			%------------------------------
			{\large $A =$ Área sob o gráfico \hspace{1cm} $A = \int_{a}^{b} f(x) \ dx$ \hspace{1cm} $A = - \int_{a}^{b} f(x) \ dx$}
			%------------------------------
			\tabularnewline
			\midrule
			%------------------------------
			\textbf{Propriedades Operatórias}
			%------------------------------
			\tabularnewline
			\midrule
			%------------------------------
			{\large \begin{tabular}[c]{@{}c@{}}
			
    			$\int \limits_{a}^{a} f(x) \ dx = F(a) - F(a) = 0$ \\
    			
    			$\int \limits_{a}^{b} f(x) \ dx = - \int \limits_{b}^{a} f(x) \ dx$ 
			
			\end{tabular}}
			%------------------------------
			\tabularnewline
			\midrule
			%------------------------------
			\textbf{Integral Imprópria}
			%------------------------------
			\tabularnewline
			\midrule
			%------------------------------
			{\large $\int \limits_{a}^{b} f(x) \ dx = \int \limits_{a}^{c} f(x) \ dx + \int \limits_{c}^{b} f(x) \ dx $ , $(a < c < b)$}
			%------------------------------
			\tabularnewline
			\bottomrule

		\end{longtable}