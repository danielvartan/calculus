\section{Função Polinomial}

	\subsection{Definição}
	
	Uma função polinomial $ f : \mathbb{R} \rightarrow \mathbb{R} $ de grau $ n $ é uma função da forma
	
	\bigskip
	
	{\Large $ y = f'(x) = a_{n} x^{n} + a_{n-1} x^{n-1} + \dots + a_{3} x^{3} + a_{2} x^{2} + a_{1} x + a_{0} $}
	
	\bigskip
	
	onde
	
	$ n $ é o grau do polinômio; \\
	$ a_{n}, a_{n-1}, \dots , a_{3}, a_{2}, a_{1}, a_{0} $ são constantes reais $ a_{n} \neq 0 $; \\
	$ x $ é a variável independente; \\
	$ y = f(x) $ é a variável dependente;
	
	\subsection{Função do 1º Grau}
	
	{\Large $ y = ax + b $}
	
	\bigskip
	
	onde $ a = m $ é o coeficiente angular e $ b $ o coeficiente linear.
	
	\bigskip
	
	{\Large $ m = \cfrac{\Delta y}{\Delta x} = \cfrac{y_2 - y_1}{x_2 - x_1}$}
	
	
	\subsection{Equação Quadrática (Fórmula de Bhaskara)}
	
	{\Large $ ax^{2} + bx + c = 0 $}
	
	\bigskip
	
	onde
	
	\bigskip
	
	{\Large $ x = \cfrac {-b \pm \sqrt {b^2 - 4ac}} {2a} $}

		\subsubsection{Discriminante da equação quadrática}
	
		{\Large $ \Delta = b^{2} - 4ac $}
		
		\bigskip
		
		Se $ \Delta > 0 $ a equação tem duas raízes reais e distintas \\
		Se $ \Delta = 0 $ a equação tem duas raízes reais e iguais (tecnicamente chamada de raiz dupla), ou popularmente "uma única raiz". \\
		Se $ \Delta < 0 $ a equação não possui qualquer raiz real.
		
		\subsubsection{Sentido da parábola}
		
		Caso $ a > 0 $ a parábola terá o aspecto de côncava para baixo (ou somente côncava). \\
		Caso $ a < 0 $ a parábola terá aspecto de côncava para cima (ou convexa).